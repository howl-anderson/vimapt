\documentclass{report}
\usepackage{listings}
\title{Vim plugin manager}
\author{Howl Anderson}
\date{\today}
\begin{document}
\maketitle
\chapter{Introduction}
Vim is a great text editor. It can use plugin to extend vim's function. But as you know vim don't have office plugin manager. Recently, Emacs: the great enemy of vim, finally has it's office plugin/package manager. But vim plugin manager is still in the war. We will study some of famous vim plugin manager, learn them how to work.
\chapter{Vundle}
Vundle is the short for Vim bundle and is a vim plugin manager.
office website is: https://github.com/gmarik/vundle
Vundle maybe was the most popular vim plugin manager on the world.
\section{Quick start}
\subsection{setup Vundle}
\begin{lstlisting}[language=bash]
$ git clone https://github.com/gmarik/vundle.git ~/.vim/bundle/vundle
\end{lstlisting}
\subsection{Configure bundles}
Sample .vimrc
\begin{lstlisting}
set nocompatible               " be iMproved
filetype off                   " required!

set rtp+=~/.vim/bundle/vundle/
call vundle#rc()

" let Vundle manage Vundle
" required! 
Bundle 'gmarik/vundle'

" My Bundles here:
"
" original repos on github
Bundle 'tpope/vim-fugitive'
Bundle 'Lokaltog/vim-easymotion'
Bundle 'rstacruz/sparkup', {'rtp': 'vim/'}
Bundle 'tpope/vim-rails.git'
" vim-scripts repos
Bundle 'L9'
Bundle 'FuzzyFinder'
" non github repos
Bundle 'git://git.wincent.com/command-t.git'
" git repos on your local machine (ie. when working on your own plugin)
Bundle 'file:///Users/gmarik/path/to/plugin'
" ...

filetype plugin indent on     " required!
"
" Brief help
" :BundleList          - list configured bundles
" :BundleInstall(!)    - install(update) bundles
" :BundleSearch(!) foo - search(or refresh cache first) for foo
" :BundleClean(!)      - confirm(or auto-approve) removal of unused bundles
"
" see :h vundle for more details or wiki for FAQ
" NOTE: comments after Bundle command are not allowed..
\end{lstlisting}
\subsection{Install configured bundles}
Launch vim, run :BundleInstall (or vim +BundleInstall +qall for CLI lovers)\\
Windows users see Vundle for Windows\\
Installing requires Git and triggers Git clone for each configured repo to ~/.vim/bundle/.
\end{document}
